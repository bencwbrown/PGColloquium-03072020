\documentclass{beamer}
\usetheme[colour=UoEblue]{UsherNew}
%% Alternative options are:
%\usetheme[colour=USHERgreen]{UsherNew}
%\usetheme[colour=USHERblue]{UsherNew}
\usepackage{beamernotes}
\usepackage{amsmath}
\pdfcompresslevel=9
\pdfobjcompresslevel=3

\title[]{Decrypting Elliptic Curves...}
\subtitle{Decrypting Elliptic Curves...}
\author{Benjamin Brown}
%\institute{Institute Name}
\date{3\textsuperscript{rd} July 2020}

\begin{document}

\begin{frame}
  \titlepage
\end{frame}
\bnote{This generates notes for pdfpc. These notes also appear
  on the handout/article versions.}

\begin{frame}[t]{Ellipses}
	After circles, ellipses are the most familiar curves in mathematics.
	\begin{figure}[h]
		\centering
		\includegraphics[width=10cm]{st-patricks-square}
		\caption{St Patrick's Square.}
	\end{figure}
\end{frame}

\begin{frame}	
	copy page over
\end{frame}

\begin{frame}[t]{Elliptic Functions}
	Arc-length of an ellipse?
\begin{equation*}
	x = a\cos\theta,\quad y=b\sin\theta\quad \quad \quad 
\end{equation*}
	
\begin{equation*}
	\begin{split}
		L &= \int_{0}^{2\pi} \sqrt{ ( \partial_{\theta}x )^{2} + ( \partial_{\theta}y )^{2} }\, d\theta \\
		&= 4a \int_{0}^{1} \sqrt{ 1 - k^{2} u^{2} }\, du \quad (k^{2} = (a^{2} - b^{2})/a^{2};\quad u = \sin\theta ) \\
		&= \frac{1}{2} \int_{1-k^{2}}^{1} \frac{t\, dt}{\sqrt{t(t-1)(t - (1-k^{2}))}}\quad (t = 1 - k^{2}u^{2})
	\end{split}
\end{equation*}	
\end{frame}

\begin{frame}
	Integrals of the form
	\begin{equation*}
		w = f(z) = \int_{P}^{z} \frac{t\, dt}{c_{3}t^{3} + c_{2}t^{2} + c_{1}t + c_{0}}
	\end{equation*}
	are called \emph{elliptic integrals (of the 2\textsuperscript{nd} kind)}. \\~\\
	
	In general, involve integrals with square-roots of cubic or quartic polynomials. \\~\\
	
	\emph{Elliptic functions} are the inverses to elliptic integrals, $z = f^{-1}(w)$. \\~\\
		
	They have no solution involving only elementary functions.
\end{frame}

\begin{frame}
	Denominator is \emph{square-root} of:
	\begin{equation*}
		y^{2} = x(x-1)(x-\lambda).
	\end{equation*}
	Over $\mathbb{C}$, it is the equation for a \emph{Legendre form} elliptic curve (if $\lambda \neq 0,1$ for distinct roots). \\~\\
	
	Problem: there is no single-valued function $\sqrt{y}$ on the whole of $\mathbb{C}$. \\~\\
	
	Integrals $\int_{\infty}^{z}\tfrac{dt}{\sqrt{y}}$ are path-dependent.
\end{frame}

\begin{frame}
	Integrals $\int_{\infty}^{z}\tfrac{dt}{\sqrt{y}}$ are path-dependent. How to get around this? \\~\\
	
	Root ``branches'' $y = \pm \sqrt{x(x-1)(x-\lambda)}$ coincide at $x = 0,1,\lambda,$ and $\infty$.	
\end{frame}

\begin{frame}
	$y = \pm \sqrt{x(x-1)(x-\lambda)}$ with branch points $x = 0,1,\lambda,\infty$. \\~\\
	
	Have one copy of $S^{2}$ for $|y|$, and another copy for $-|y|$. \\~\\
	
	We make cuts between $\infty \longleftrightarrow \lambda$ and $1 \longleftrightarrow 0$ on each $S^{2}$, and glue them together.
	
\end{frame}

\begin{frame}
\end{frame}

\begin{frame}
\end{frame}

\begin{frame}
\end{frame}

\begin{frame}
\end{frame}



\end{document}
